\documentclass[11pt]{article}
\usepackage{latexsym}
\usepackage{float}
\usepackage{fullpage}
\usepackage{amsfonts}
\usepackage{amsmath, amsthm, amssymb}
\usepackage{graphicx}
\usepackage{endnotes}
\usepackage{epsfig}
\usepackage{setspace}
\usepackage{natbib}
\usepackage{multirow}
\usepackage[hang]{subfigure}
\parskip=6pt

\newcommand{\latex}{\LaTeX}
\newcommand{\be}{\begin{enumerate}}
\newcommand{\ee}{\end{enumerate}}
\newcommand{\bq}{\begin{quote}}
\newcommand{\eq}{\end{quote}}
\newcommand{\bd}{\begin{description}}
\newcommand{\ed}{\end{description}}
\newcommand{\bi}{\begin{itemize}}
\newcommand{\ei}{\end{itemize}}
\newcommand{\brr}{\begin{raggedright}}
\newcommand{\err}{\end{raggedright}}
\newcommand{\bl}{\begin{Large}}
\newcommand{\el}{\end{Large}}
\newcommand{\eg}{for example}
\renewcommand{\refname}{\normalsize\centering References}
\hyphenpenalty=5000
\exhyphenpenalty=4000
\tolerance=2000

\singlespace
\begin{document}
\small
\begin{centering} \textbf {POLS6382 Quantitative Methods III: Maximum Likelihood Estimation}\\

\vspace{2mm}
Ling Zhu \\
Department of Political Science\\
University of Houston \\
October 4, 2021 \\
\vspace{2mm}
 \textbf{Lab 5:  Models for Ordinal Data}\\
\end{centering}

\section*{\normalsize Objectives}
\begin{itemize} 
\item Learn how to estimate a model suitable for ordinal outcomes.
\item Learn how to substantively interpret the statistical results of an ordered logit model. 
%\item Learn how to test for the parallel regression assumption.
%\item Learn how to estimate a multinomial logit model and substantively interpret the statistical results.

\end{itemize}

\section{\normalsize Data Example: Americans' Preferences on Immigrants' Access to Public Health Care}
In this section, we will learn how to estimate an ordered logistic regression model, using the example data, \texttt{gss2012lab5.dta}. Data are drawn from the 2012 General Social Survey (GSS). In the datafile, we have the following variables. For more details about the dataset, see Zhu, Ling and Kenicia Wright. 2016. ``Why Do Americans Dislike Publicly Funded Health Care? Examining the Intersection of Race and Gender in the Ideological Context." \textit{Politics, Groups, and Identities}, 4(4): 618-637.
\begin{itemize}
\item \texttt{hlthctzen}: Respondents' answers to the question: \textit{Do you agree with the following statement?One should have access to public funded health care if he/she is not a citizen. } This variable is coded as three categories: ``1"= disagree, ``2"=neigher, and ``3"= agree with the statement. 
\item \texttt{female}: respondents' gender, coded as ``1" for female respondents, and ``0" otherwise. 
\item \texttt{white}: Respondents' race, coded as ``1" for whites and ``0" otherwise.
\item \texttt{age}: Respondents' age.
\item \texttt{educ}: Year of eduction.
\item \texttt{income}: measured as levels: 1=1-\$1,000, 2=$1,00-2,999, 3=$3,000-3,999....12=25,000 or more. 
\item \texttt{partyid}: party identification, measured as levels: 0= strong Democrat, 1=not strong Democrat, 2= independent, near democrat, 3=independent, 4=independent, near Republican, 5=not strong Republican, 6=strong Republican.
\item \texttt{ideology}: liberal-conservative ideology scale, coded as a 1-to-7 scale, reflecting conservatism. 
\end{itemize}


The variable of interest is \texttt{hlthctzen}, respondents' opinions about whether non-citizens should have access to public health care or not. We may just use some descriptive statistics to learn about the sample distribution. A histogram plot could be useful to describe how preferences vary. From Figure 1 we see that the total sample size is about 1,300 cases. About 800 respondents disagree with the statement that a non-citizen should have access to public health care. About 400 respondents agreed with the statement.

\noindent------------------------------------- R Code-------------------------------------------
\small
\begin{verbatim}
pdf(file="response.pdf",height=6, width=6)
hist(hlthctzen,
     main="Access to public funded health care if one is not a citizen?",
     xlab="Responses"
     )
dev.off()
\end{verbatim}
\noindent------------------------------------- R Output-------------------------------------------
\vspace{-4mm}
\begin{figure}[H]
\caption{Public Preferences on Non-Citizens' Access to Public Health Care }
\includegraphics[height=5in,width=5in]{response.pdf} 
\end{figure}

We can also obtain marginal proportions for each choice category using \texttt{R} function, \texttt{prob. table()}. Based on the 2012 GSS, 60.6\% of the respondents disagreed with the statement.  About 12\% of the respondents reported neutral attitudes, and only 27.3\% of the respondents agreed that non-citizens should have access to public health care. In other words, the majority opinion is that non-citizens should be excluded from accessing public health care. 

\noindent------------------------------------- R Code/Output-------------------------------------------
\small
\begin{verbatim}
> prop.table(table(hlthctzen))
hlthctzen
        1         2         3 
0.6056231 0.1215805 0.2727964 

\end{verbatim}

\section{\normalsize Estimating an Ordered Logit Model with \texttt{polr()}}
Various \texttt{R} functions can be used to estimate an ordered logit/probit model. One commonly used function is \texttt{polr()} from package \texttt{MASS}. Estimation with \texttt{polr()} is simple. We define a model equation, and the datafile. The default link function is ``logit". The statement \texttt{Hess=TRUE} returns the Hessian matrix in model object, which will be used in post-estimation simulation later. We specify our ordered-logit model by including \texttt{gender, race, age, education, income,} and \texttt{ideology} as the explanatory variables. 

\noindent------------------------------------- R Code/Output-------------------------------------------
\begin{verbatim}
> model1<-polr(as.factor(hlthctzen)~female+white+age+educ+income+ideology,
+              data=healthcare,Hess=TRUE)
> summary(model1)
Call:
polr(formula = as.factor(hlthctzen) ~ female + white + age + 
    educ + income + ideology, data = healthcare, Hess = TRUE, 
    model = TRUE)

Coefficients:
              Value Std. Error t value
female    0.0786593   0.118059  0.6663
white    -1.0827232   0.135465 -7.9926
age      -0.0009609   0.003554 -0.2704
educ      0.0358809   0.020647  1.7378
income   -0.0156195   0.029821 -0.5238
ideology -0.3705436   0.042046 -8.8129

Intercepts:
    Value   Std. Error t value
1|2 -1.5645  0.4641    -3.3707
2|3 -0.9415  0.4631    -2.0332

Residual Deviance: 2232.765 
AIC: 2248.765 
\end{verbatim}

You may notice from the above \texttt{R} output that \texttt{polr()} does not return p-values. We can look at the t-value column to evaluate if a variable has a significant effect on respondents' preferences regarding non-citizen's access to health care. You can also get p-values using \texttt{pnorm()}. Using \texttt{stargazer()}, we can convert the R-output in a well-arranged table in \LaTeX\ .

\noindent------------------------------------- R Code/Output-------------------------------------------
\vspace{-4mm}
\begin{verbatim}
> # get p values
> p<-pnorm(abs(ctable[, "t value"]), lower.tail = FALSE) * 2
> #combined table
> (ctable <- cbind(ctable, "p value" = p))
                 Value Std. Error    t value      p value
female    0.0786593115 0.11805945  0.6662687 5.052394e-01
white    -1.0827231790 0.13546543 -7.9926159 1.321053e-15
age      -0.0009608995 0.00355360 -0.2704017 7.868512e-01
educ      0.0358808677 0.02064692  1.7378316 8.224049e-02
income   -0.0156194898 0.02982057 -0.5237824 6.004299e-01
ideology -0.3705435895 0.04204551 -8.8129176 1.219302e-18
1|2      -1.5644559949 0.46414069 -3.3706504 7.499097e-04
2|3      -0.9415444911 0.46307425 -2.0332474 4.202754e-02
\end{verbatim}

\begin{table}[H] \centering 
  \caption{The Determinants of Preferences on Non-Citizen's Access to Public Health Care} 
  \label{} 
\begin{tabular}{@{\extracolsep{5pt}} lllllll} 
\\[-1.8ex]\hline 
\hline \\[-1.8ex] 
 & Value & Std. Error & t value & p value \\ 
\hline \\[-1.8ex] 
female & $0.079$ & $0.118$ & $0.666$ & $0.505$ \\ 
white & $$-$1.083$ & $0.135$ & $$-$7.993$ & $0.000$ \\ 
age & $$-$0.001$ & $0.004$ & $$-$0.270$ & $0.787$ \\ 
education & $0.036$ & $0.021$ & $1.738$ & $0.082$ \\ 
income & $$-$0.016$ & $0.030$ & $$-$0.524$ & $0.600$ \\ 
ideology & $$-$0.371$ & $0.042$ & $$-$8.813$ & $0.000$ \\\hline 
$\tau_{1}$ & $$-$1.564$ & $0.464$ & $$-$3.371$ & $0.001$ \\ 
$\tau_{2}$& $$-$0.942$ & $0.463$ & $$-$2.033$ & $0.042$ \\ 
\hline \\[-1.8ex] 
\end{tabular} 
\end{table} 

\section{\normalsize Substantive Interpretation}
How can we substantively interpret the coefficients reported in Table 1? Scott Long discussed various methods in his textbook. Here, we will focus on using odds ratios and predicted probabilities. Because we estimated an ordered logit model, we can convert coefficients into odds ratios, using function \texttt{exp()}. Combining \texttt{exp()} with \texttt{confint()}, we can also obtain confidence intervals for odds ratios. If the confidence intervals of an odds ratio go across 1, it means that the odds ratio coefficient is not statistically significant. 


\noindent------------------------------------- R Code/Output-------------------------------------------
\small
\begin{verbatim}
> exp(coef(model1))
   female     white       age      educ    income  ideology 
1.0818357 0.3386720 0.9990396 1.0365324 0.9845019 0.6903590 
\end{verbatim}

We observe that \texttt{white} and \texttt{ideology} significantly affect respondents' preferences regarding non-citizens' access to public health care. 
\begin{enumerate}
\item The odds ratio associated with \texttt{white} is 0.34: the odds of reporting ``neither" and ``agree" versus ``disagree" are 66 percent lower for white respondents than that for non-white respondents. In other words, compared with non-white respondents, white respondents are more likely to choose ``disagree" than ``neither" and ``agree". 

\item \texttt{ideology} has an odds ratio of 0.69: a one-point increase in conservatism decreases the probability of choosing ``agree" over ``neither" and ``disagree" by about 31\%. In other words, moving from liberalism to conservatism will decrease the probability of choosing ``agree" over the other two lower categories. 
\end{enumerate}

Beyond using odds ratios, another useful way to substantively interpret findings is to calculate predicted probabilities associated with each choice category. First, we will discuss how to present mean predictions. Suppose that we are interested in gauging the substantive effect of \texttt{ideology} and \texttt{race} (being white) on preferences. What we can do is to predict the probabilities of choosing each choice category across the full range of the ideology scale, but setting the dummy variable \texttt{white} to be 1 and 0. This would give us two sets of predictions: one for white respondents, and one for non-white respondents. Note that we hold all the other variables constant. \texttt{female=mean(female)} essentially gives variable \texttt{female} an artificial value. This is to average the weights between female and male respondents in the sample. Alternatively, you can set \texttt{female} to be 0 or 1. That will give you the comparison between white male and non-white male respondents (or white and non-white female respondents) across the full range of \texttt{ideology}.

\noindent------------------------------------- R Code-------------------------------------------
\begin{verbatim}
# Mean predicted probabilities, white respondents
predictdata<-cbind(ideology=seq(1,7,length=100),
                   age=mean(age),income=mean(income),educ=mean(educ),
                   female=mean(female),white=1)
opinion.hat<-predict(model1,predictdata,type='prob')

# Mean predicted probabilities, non-white respondents
predictdata2<-cbind(ideology=seq(1,7,length=100),
                   age=mean(age),income=mean(income),educ=mean(educ),
                   female=mean(female),white=0)
opinion.hat2<-predict(model1,predictdata2,type='prob')

# Plot pps
ideology<-seq(1,7,length=100)
pdf(file="pp_white.pdf",height=7, width=7)
plot(c(1,7),c(0,1),type='n',
     xlab="Liberal-Conservative Ideology Scale",
     ylab="Predicted Probabilities (y=j)",
     main="Access to public funded health care if one is not a citizen?")
lines(ideology,opinion.hat[1:100,1],lty=1,lwd=3,col="red")
lines(ideology,opinion.hat[1:100,2],lty=2,lwd=3,col="blue")
lines(ideology,opinion.hat[1:100,3],lty=3,lwd=3,col="green")
legend(1,1,cex=0.9,c('Disagree','Neither','Agree'),
       lty=1:3,col=c("red","blue","green"))
dev.off()

pdf(file="pp_nonwhite.pdf",height=7, width=7)
plot(c(1,7),c(0,1),type='n',
     xlab="Liberal-Conservative Ideology Scale",
     ylab="Predicted Probabilities (y=j)",
     main="Access to public funded health care if one is not a citizen?")
lines(ideology,opinion.hat2[1:100,1],lty=1,lwd=3,col="red")
lines(ideology,opinion.hat2[1:100,2],lty=2,lwd=3,col="blue")
lines(ideology,opinion.hat2[1:100,3],lty=3,lwd=3,col="green")
legend(1,1,cex=0.9,c('Disagree','Neither','Agree'),
       lty=1:3,col=c("red","blue","green"))
dev.off()
\end{verbatim}
\noindent------------------------------------- R Output-------------------------------------------
\begin{figure}[H]
\caption{Respondents' Preferences on Non-Citizens' Access to Public Health Care  by Ideology and Race }
$\begin{array}{ll}
\subfigure[White Respondents]{\includegraphics[scale=0.5]{pp_white.pdf}}&
\subfigure[Non-White Respondents]{\includegraphics[scale=0.5]{pp_nonwhite.pdf}}
\end{array}$
\end{figure}

Figure 2 shows \texttt{ideology} has a negative impact on supporting non-citizens' access to public health care, which is consistent across the two racial groups. For white respondents, conservatism increases the predicted probabilities of choosing ``disagree" with non-citizens having access to public health care, while substantially decreases the predicted probability of choosing ``agree" with non-citizens having access to public health care. A similar pattern is observed for non-white respondents that conservatism increases the probability of choosing ``disagree" while decreases the probability of choosing ``agree".

To show the substantive effect of race, we can graph white and non-white respondents' preferences in one figure. Because an ordered logit/probit model will produce predicted probabilities for at least three categories, plotting all these predicted probabilities in one figure is not a good choice. what we can do is to compare white and non-white respondents' predicted probabilities by choice category. In practice, we normally focus on the bottom and top categories. 

\noindent------------------------------------- R Code-------------------------------------------
\begin{verbatim}
# Compare white and non-white respondents' PPs of "disagree"
pdf(file="compare.pdf",height=7, width=7)
plot(c(1,7),c(0,1),type='n',
     xlab="Liberal-Conservative Ideology Scale",
     ylab="Predicted Probabilities (y=Disagree)",
     main="Access to public funded health care if one is not a citizen?")
lines(ideology,opinion.hat[1:100,1],lty=1,lwd=3,col="red")
lines(ideology,opinion.hat2[1:100,1],lty=2,lwd=3,col="blue")
legend(1,1,cex=0.9,c('White','Non-White'),
       lty=1:3,col=c("red","blue"))
dev.off()

# Compare whit and non-white respondents' PPs of "agree"
pdf(file="compare2.pdf",height=7, width=7)
plot(c(1,7),c(0,1),type='n',
     xlab="Liberal-Conservative Ideology Scale",
     ylab="Predicted Probabilities (y=Agree)",
     main="Access to public funded health care if R is not a citizen?")
lines(ideology,opinion.hat[1:100,3],lty=1,lwd=3,col="red")
lines(ideology,opinion.hat2[1:100,3],lty=2,lwd=3,col="blue")
legend(1,1,cex=0.9,c('White','Non-White'),
       lty=1:3,col=c("red","blue"))
dev.off()
\end{verbatim}
\noindent------------------------------------- R Output-------------------------------------------
\begin{figure}[H]
\caption{Comparing White and Non-White Respondents' Preferences on Non-Citizens' Access to Public Health Care }
$\begin{array}{ll}
\subfigure[White Respondents]{\includegraphics[scale=0.5]{compare.pdf}}&
\subfigure[Non-White Respondents]{\includegraphics[scale=0.5]{compare2.pdf}}
\end{array}$
\end{figure}

Figure 3 shows, across the full range of ideology, white respondents are associated with higher predicted probabilities of choosing ``disagree" than non-white respondents; and lower probabilities of choosing ``agree" than non-white respondents. Figure 3 shows salient racial differences in attitudes toward non-citizens' access to public health care. 


\end{document}
\section{\normalsize Data Example 2: 1996 American National Election Study (ANES)}
In this section, we will learn how to assess the parallel regression assumption, one that we make when specifying an ordered logistic regression model. Consider an example used in Faraway Chapter 5, a subset of the 1996 ANES. In the dataset, we only consider a few variables: age, income, and education level. The response variable of interest will be the party identification of respondents. We recode several variables in the following way. 

\begin{itemize}
\item \texttt{party}: We generate a new variable, \texttt{party} by collapsing the original ANES variable \texttt{PID} from five categories into three categories.
\item \texttt{income}: The original ANES income variable was an ordered factor with income ranges. We convert this to a numeric variable by taking the midpoint of each range. 
\item We then create a new data frame and name the datafile as \texttt{rnes96}
\end{itemize} 

\section{\normalsize Preliminary Exploration}

We begin by taking a look at the data. The responses are coded at the individual level, so we want to describe the observed proportions of the three categories (the recoded party id variable) by the predictors (e.g. income, education, etc.). Using package \texttt{dplyr}, we group data by education and income levels, then compute the proportion supporting each party for each party affiliation. The resulting descriptive plots are presented in Figure 1. We observe that the proportion of Democrats falls with education level, and the proportion of Republicans increases with education level. A similar pattern is observed regarding the association between responses and income. 


\noindent------------------------------------- R Code-------------------------------------------
\small
\begin{verbatim}
# Plot proportions of choices
library(dplyr)
# Education and Party ID#
egp <- group_by(rnes96, education, party) %>% summarise(count=n()) %>% 
	group_by(education) %>% mutate(etotal=sum(count), proportion=count/etotal)
require(ggplot2)
ggplot(egp, aes(x=education, y=proportion, group=party, linetype=party))+
  geom_line()+theme_bw()+
  labs(x="Education Attainment",
  y="Observed Sample Proportions")+
  theme(axis.text = element_text(size = 10), 
        axis.title=element_text(size= 12),
        plot.margin = unit(c(0.5,0.5,0.5,0.5), "cm"))

# Income and Party ID
igp <- mutate(rnes96, incomegp=cut_number(income,7)) %>% group_by(incomegp, party) %>%
	 summarise(count=n()) %>% group_by(incomegp) %>% mutate(etotal=sum(count), proportion=count/etotal)

ggplot(igp, aes(x=incomegp, y=proportion, group=party, linetype=party))+
  geom_line()+theme_bw()+
  labs(x="Income Levels",
       y="Observed Sample Proportions")+
   theme(axis.text = element_text(size = 10,angle=90), 
        axis.title=element_text(size= 12),
        plot.margin = unit(c(0.5,0.5,0.5,0.5), "cm"))
        
\end{verbatim}
\noindent------------------------------------- R Output-------------------------------------------
\begin{figure}[H]
\caption{Associations between Party Affiliation, Education, and Income}
$\begin{tabular}{ll}
\includegraphics[height=3in,width=3.5in]{educcpartyid.pdf} &
\includegraphics[height=3in,width=3.5in]{incpartyid.pdf} 
\end{tabular}$
\end{figure}

\section{\normalsize Testing the Parallel Regression Assumption (Proportional Odds Assumption)}
Next, we consider various ways by which we can evaluate the parallel regression assumption. If we estimate an ordered logit regression model for the party affiliation variable, we need to evaluate if the parallel regression assumption holds or not. First we can assess the parallel regression assumption in a ``crude way" by examining if observed odds proportions with respect to the predictor (e.g. income) are constant. Using the function \texttt{prop.table}, we obtain odds proportions with respect to income levels. We then compute log-odds differences between $\gamma_{1}$ and $\gamma_{2}$ ( the two cumulative probability areas). We observe that these log-odds differences are not constant across different income levels, thus have some evidence for the violation of the proportional odds assumption. 

\noindent------------------------------------- R Code-------------------------------------------
\small
\begin{verbatim}
> pim<-prop.table(table(income, party), 1)
> logit(pim[,1])-logit(pim[,1]+pim[,2])
       1.5          4          6          8        9.5       10.5 
-0.9007865 -2.0614230 -0.7576857 -1.0033021 -2.3025851 -0.3083014 
      11.5       12.5       13.5       14.5         16       18.5 
-0.7985077 -1.8971200 -1.2527630 -1.1786550 -0.4128452 -0.3542428 
        21       23.5       27.5       32.5       37.5       42.5 
-1.5141277 -1.6534548 -0.7467847 -0.5225217 -0.9232594 -1.0296194 
      47.5         55       67.5       82.5       97.5        115 
-0.8219801 -1.4276009 -1.1826099 -0.9867640 -1.4829212 -1.7066017 
\end{verbatim}

We can also evaluate this assumption using the Brant Test. After fitting an ordered logit model, we can use function \texttt{brant} to test the parallel regression assumption. The Null hypothesis is that the parallel regression assumption holds. Therefore, we see from the Brant Test that variable \texttt{income} violates the parallel regression assumption, producing the statistically significant "Omnibus" Chi-squared statistics.

\noindent------------------------------------- R Code-------------------------------------------
\small
\begin{verbatim}
> library(MASS)
> pomod<-polr(party~age+education+income, rnes96); summary(pomod)
Call:
polr(formula = party ~ age + education + income, data = rnes96)

Coefficients:
                Value Std. Error  t value
age          0.005775   0.003887  1.48581
education.L  0.724087   0.384388  1.88374
education.Q -0.781361   0.351173 -2.22500
education.C  0.040168   0.291762  0.13767
education^4 -0.019925   0.232429 -0.08573
education^5 -0.079413   0.191533 -0.41462
education^6 -0.061104   0.157747 -0.38735
income       0.012739   0.002140  5.95187

Intercepts:
                       Value   Std. Error t value
Democrat|Independent    0.6449  0.2435     2.6479
Independent|Republican  1.7374  0.2493     6.9694

Residual Deviance: 1984.211 
AIC: 2004.211 

require{brant}        
 brant(pomod)
-------------------------------------------- 
Test for	X2	df	probability 
-------------------------------------------- 
Omnibus		27.6	8	0
age		0.74	1	0.39
education.L	-4.73	1	1
education.Q	-3.56	1	1
education.C	-0.01	1	1
education^4	0	1	1
education^5	-0.06	1	1
education^6	-0.03	1	1
income		7.99	1	0
--------------------------------------------
\end{verbatim}

We can also use the cumulative link model to evaluate the parallel regression assumption. Using the function \texttt{clm()} from package \texttt{ordinal}, we firs define the base cumulative link model. Then we incorporate nominal effects in a different model. Statement \texttt{nominal=~+income} is used to specify that income is associated with nominal effects. Next, we perform Likelihood ratio test for the base model and the model with nominal effects. Significant chi-square
indicates violation of the assumption.  Using this alternative method, we reach to the same conclusion that there are nominal effects associated with income. In other words, variable \textit{income} violates the parallel regression assumption. 

\begin{verbatim}
require(ordinal)
# Define the base cumulative link model
base<-clm(party~age+educ+income, data=rnes96)

# Incoporating nominal effects
nom.mod<-clm(party~age+educ, nominal=~+income, data=rnes96)

anova(base, nom.mod)

> nom.mod<-clm(party~age+education, nominal=~+income, data=rnes96)
> anova(base, nom.mod)
Likelihood ratio tests of cumulative link models:
 
        formula:                         nominal: link: threshold:
base    party ~ age + education + income ~1       logit flexible  
nom.mod party ~ age + education          ~+income logit flexible  

        no.par    AIC  logLik LR.stat df Pr(>Chisq)   
base        10 2004.2 -992.11                         
nom.mod     11 1998.7 -988.33  7.5519  1   0.005995 **
---
\end{verbatim}

\section{\normalsize Estimating an Multinomial Logit Regression}

When modeling multiple-choice responses, where by the parallel regression assumption is violated. The multinomial logit specification is better than the ordered logit specification. Substantively, multinomial logit regression treats choice categories as nominal responses without assumption that  these choices are ordered. A multinomial logit model can be estimated using function \texttt{multinom} or function \texttt{vglm}. Function \texttt{mlogit} in R also supports the estimation of an ordered logit model. \\

\noindent------------------------------------- R Code-------------------------------------------
\begin{verbatim}
library(nnet)
mmod<-multinom(party~age+education+income, rnes96)
summary(mmod)
\end{verbatim}

Table 2 shows the results from the multinomial logit regression model, with the first choice category ("Democrat") being the omitted baseline category. The coefficient of variable "age" is insignificant, meaning that age does not increase one's probability of identifying as an independent compared with that for identifying with the Democratic party. In the second equation, variable age has a positive and significant coefficient, meaning that age is positively associated with the probability of identifying with the Republican party, compared with that for identifying with the Democratic party.

\begin{table}[!htbp] \centering 
  \caption{Multinomial Logit Regression for Party Identification} 
  \label{} 
\begin{tabular}{@{\extracolsep{5pt}}lcc} 
\\[-1.8ex]\hline 
\hline \\[-1.8ex] 
 & \multicolumn{2}{c}{\textit{Dependent variable:}} \\ 
\cline{2-3} 
\\[-1.8ex] & Independent & Republican \\ 
\\[-1.8ex] & (1) & (2)\\ 
\hline \\[-1.8ex] 
 age & 0.0002 & 0.008$^{*}$ \\ 
  & (0.005) & (0.005) \\ 
  & & \\ 
 education.L & 0.064 & 1.194$^{*}$ \\ 
  & (0.457) & (0.650) \\ 
  & & \\ 
 education.Q & $-$0.122 & $-$1.229$^{**}$ \\ 
  & (0.414) & (0.604) \\ 
  & & \\ 
 education.C & 0.112 & 0.154 \\ 
  & (0.350) & (0.487) \\ 
  & & \\ 
 education$\hat{\mkern6mu}$4 & $-$0.077 & $-$0.028 \\ 
  & (0.288) & (0.361) \\ 
  & & \\ 
 education$\hat{\mkern6mu}$5 & 0.136 & $-$0.122 \\ 
  & (0.249) & (0.270) \\ 
  & & \\ 
 education$\hat{\mkern6mu}$6 & 0.154 & $-$0.037 \\ 
  & (0.217) & (0.203) \\ 
  & & \\ 
 income & 0.016$^{***}$ & 0.017$^{***}$ \\ 
  & (0.003) & (0.003) \\ 
  & & \\ 
 Constant & $-$1.197$^{***}$ & $-$1.643$^{***}$ \\ 
  & (0.327) & (0.331) \\ 
  & & \\ 
\hline \\[-1.8ex] 
Akaike Inf. Crit. & 2,004.333 & 2,004.333 \\ 
\hline 
\hline \\[-1.8ex] 
\textit{Note:}  & \multicolumn{2}{r}{$^{*}$p$<$0.1; $^{**}$p$<$0.05; $^{***}$p$<$0.01} \\ 
\end{tabular} 
\end{table} 


We can also use predicated probabilities to visualize key results. The following R code shows predicted probabilities for each choice category along the income variable, holding other variables constant. 
\noindent------------------------------------- R Code-------------------------------------------
\begin{verbatim}
mmodi<-step(mmod) #Shows that model with lowest AIC has only income
inclevels <- 0:110
preds <- data.frame(income=inclevels,
          predict(mmodi,data.frame(income=inclevels),type="probs"))
library(tidyr)
lpred <- gather(preds, party, probability, -income)
require(ggplot2)
pdf(file="multinomialpp2.pdf", width=7, height=5)
ggplot(lpred, aes(x=income,y=probability,group=party,linetype=party))+geom_line()+theme_bw()
dev.off()
\end{verbatim}

\noindent------------------------------------- R Output-------------------------------------------
\vspace{-4mm}
\begin{figure}[H]
\caption{Party Identification and Income }
\includegraphics[height=4in,width=5.5in]{multinomialpp2.pdf} 
\end{figure}
\end{document}
